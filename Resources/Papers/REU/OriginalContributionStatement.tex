\documentclass[12pt]{article}

\title{Personal Contribution Statement \\\large For Laser Frequency Stabilization Project at the University of Kentucky REU Program}
\author{Will Bodron}

\begin{document}
  \maketitle
  \paragraph{} I was shown, at the beginning of this project, an empty table, a ECDL infrared laser, a computer, and a storage room, and given the task of reducing the frequency drift as much as possible. Over the following weeks, I would design several disparate experiments to quantify the rate of spectral drift and to identify qualities of the D2 transition of alkali metals. Even going against experimental designs found in existing papers, I chose to leverage polarization rotation to achieve effective laser counter-propagation through a gas cell, which improved our saturated absorption signal significantly. 
  \paragraph{} Aside from constructing the optical experiments, I also took on the creation of a set of modular programs, which can be called programmatically from a single script to control all components of the experiment. Commands can be issued to the laser, temperature controller, wavemeter, and oscilloscope using easy to understand function calls to collect project specific or generic data and to set time dependent parameter control. The result is a reusable, expandable, documented, DAQ system that will see use in the Diode Laser's and partnering measurement devices' continued use on other experiments at the University.
\end{document}